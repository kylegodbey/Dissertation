\documentclass[12pt,letterpaper,doublespace]{article}
\usepackage[margin=1in]{geometry}
\usepackage[english]{babel}
\usepackage[utf8]{inputenc}
\usepackage{fancyhdr}
%\usepackage{setspace}
%\doublespacing
\renewcommand{\headrulewidth}{0pt}
\pagestyle{fancy}

\fancyhead[R]{\underline{PHYSICS}}
\pagenumbering{gobble}
%opening
\title{\normalsize LOW-ENERGY NUCLEAR REACTIONS USING TIME-DEPENDENT DENSITY FUNCTIONAL THEORY}
\author{\normalsize Kyle Godbey}
\date{\normalsize\underline{\textbf{Dissertation under the direction of Professor Sait Umar}}}


\begin{document}

\maketitle
\thispagestyle{fancy}

\openup 1em

Nuclei are small many-body quantum systems which account for most of the currently observable mass of the universe.
This abundance suggests that the study of interactions between atomic nuclei can enhance understanding of physics at all scales, despite the relatively short distance of the nuclear force.
As systems of nuclei can have several hundreds of individual particles, the mean-field approximation is often invoked to make calculations of nuclear reactions numerically tractable.
A number of investigations have been performed relating to nuclear fusion and quasifission, as well as the development of new tools and techniques to aid in the study of nuclear and quantum many-body physics.
A significant focus has been placed on the investigation of transfer and how affects nuclear reactions.
Specifically, transfer is found to substantially enhance sub-barrier nuclear fusion, and deformed shell effects are found to drive the total transfer seen in quasifission reactions.
Furthermore, transfer between symmetric systems is investigated and offers a promising path towards the creation of neutron-rich nuclei via multinucleon transfer processes.




\vspace{120pt}
\noindent\begin{tabular}{ll}
	% Member 1
	\makebox[4.0in]{Approved \hrulefill} & \makebox[1.5in]{Date \hrulefill}\\
	\hspace{130px}Sait Umar, Ph.D.
\end{tabular}
\end{document}
