% This is an example for a `Introduction'.
% To generate the final document, run latex, build and quick build commands
% on the skeleton-thesis file not this one.

\chapter{Introduction}\label{chapters:Introduction}
\vspace{-7mm}

From all scales ...
Nuclear interactions affect all

infinite nuclear matter
finite nuclei and their structure
static systems rare in nature, firm understanding of dynamics and reactions

Broad picture of dynamics of nuclear systems
focus on reactions and collisions between two nuclei
Studies can be broadly separated into time scales of reactions
though in practice, experimental studies include events from all

sec1 time scales
sec2 methods to study reactions

The specific approaches to each project are presented in their respective chapters, though a general, brief word should be said about the mean-field method as it applies to atomic nuclei.
As the nucleus is traditionally thought of as a densely packed collection of nucleons, the primary assumption that individual protons and neutrons can travel freely in the nucleus is a bit counter intuitive.
It turns out, however, that the Pauli exclusion principle between nucleons ensures that the particles will remain at an average distance exceeding their radius at low energies~\cite{ring1980}.
Even at finite collision energies, the mean free path of nucleons in the nucleus is several times that of the nuclear radius, implying that a so-called hard-core collision is unlikely.
Should one go to higher energies, the mean-field approximation begins to break down and may misrepresent the outcome of such reactions.


%% Section
\section{Section 1}\label{sec:ch_1_sec_1}

%% Subsection
\subsection{Subsection 1}\label{subsec:subsec_1.1.1}


\clearpage
