% This is an example of the `Future work' section.
% To generate the final document, run latex, build and quick build commands
% on the skeleton-thesis file (not this one)


%%
%% Final conclusion of the dissertation
\section{Conclusion}
\label{sec:future_work_conclusion}
The uniting tool behind the work presented in this thesis has been the use of TDDFT to explore nuclear reactions at the mean-field level.
From quasielastic scattering to the total fusion of nuclear fragments, TDDFT alone can readily describe the outcome of nuclear collisions.
As mentioned before, at the base level, TDHF is optimized for the description of one-body observables~\citep{balian1981}.
This predictive ability is furthered by the development and use of extensions to the base theory to uncover correlations and effects beyond the mean-field.
It is through this effort that nuclear DFT has become the dominant tool in recent years to study nuclear reactions at low energies -- energy scales that are of interest to superheavy and neutron rich element formation, seed reactions of the r-process, and even the general description of equilibration in interacting quantum many-body systems.

The sum total of this work and all others in this area serves as the foundation for the next step in studying the nuclear many-body problem as it relates to reactions and structure studies.
Through further development of extensions to the base theory and as of yet unimagined approaches to better handle many-body correlations, the future of low-energy physics relies on improving our collective understanding of how systems of many particles interact with each other.
Indeed, if an end goal could be identified it would be with the complete quantum description of many-body tunneling in fission and reactions.
While the work presented in this thesis has made steps in this direction by better describing fusion and transfer mechanisms, there is still much ground to cover.